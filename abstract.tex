\chapter{Abstract}

Unlike the terrestrial magnetosphere, which responds strongly to changes in the solar wind and interplanetary magnetic field, Jupiter's magnetosphere depends largely on internal processes related to its strong planetary magnetic field, rapid rotation and presence of internal sources of plasma. The role of external factors, like the solar wind, is considered to be minimal and is not well understood. Observations of the Jovian UV aurora have hinted that it responds to changes in the upstream solar wind dynamic pressure, though the physical processes leading to this enhancement are poorly understood. Solar wind dynamic pressure is also considered to influence the oscillations of Jupiter's current sheet, particularly in the magnetotail. These questions are difficult to resolve using limited and sparse in situ observations. 

In this work, we use a global magnetohydrodynamics model to study the dynamics of the Jovian magnetosphere. By performing systematic experiments in which we vary the solar wind dynamic pressure, we have analyzed the response of the Jovian magnetosphere and tracked the change in the intensity of corotation-enforcement currents, which are believed to be closely related to the Jovian aurora. Our model predicts that intensity of field-aligned currents in the dayside ionosphere would decrease after a dynamic pressure enhancement due to an increase in the corotation velocity of the plasma in the dayside magnetosphere, especially in the pre-noon sectors. Our model also predicts the release of plasmoids due to reconnection in the magnetotail, and we demonstrate that these plasmoids magnetically connect to the polar regions of the planet, which predominantly connect to open field lines in the solar wind. 

We also study the response of Jupiter's oscillating magnetotail current sheet to changes in the upstream solar wind dynamic pressure by incorporating a realistic internal field model into our simulations. It was previously thought that solar wind driven compression would reduce the amplitude of the current sheet oscillations and change the hinging distance of the current sheet. Using the MHD model, we find that increasing solar wind dynamic pressure also increases the density and temperature in the magnetotail lobes such that the Alfven and magnetosonic speeds in the magnetosphere may be reduced, which can decrease the wavelength of current sheet oscillations.

Lastly, our MHD model and other in situ observations have revealed that plasmoids produced in the Jovian magnetotail are large and infrequent. We have analyzed high temporal resolution magnetometer data from the Juno spacecraft and identified magnetic flux ropes and O-lines with durations lasting less than 300 seconds, which correspond to diameters comparable with the local ion-inertial length. These findings suggest that despite many differences, magnetic reconnection in the Jovian magnetotail can also operate via current sheet instabilities, similar to that seen at Earth, Mercury and in particle-in-cell simulations.

Our findings illustrate the complicated nature of dynamics in the Jovian magnetosphere, which is not well understood due to the sparse in situ data. We demonstrate that, in the absence of global empirical data, numerical experiments can serve to validate or invalidate theories of magnetospheric dynamics. Our results use and support data gathered by numerous in situ spacecraft such as Galileo and Juno, and can also support future missions to the Jupiter system. The Jovian magnetosphere is an extreme environment, and an excellent laboratory for plasma research, and additional observations and models are needed to understand its dynamics. 


