\chapter{Abstract}

Unlike the terrestrial magnetosphere, which responds strongly to changes in the solar wind and interplanetary magnetic field, Jupiter's magnetosphere depends more on internal processes related to its strong magnetic field, rapid rotation and presence of internal sources of plasma. Some have argued that magnetic reconnection between the internal field and the solar wind is unimportant, leading to a magnetosphere which is largely closed, or shielded from the direct influence of the solar wind plasma. However, the solar wind does influence the size and configuration of Jupiter's magnetosphere. Observations of the Jovian UV aurora have hinted that it responds to changes in the upstream solar wind dynamic pressure, though the physical process leading to this enhancement are poorly understood. Solar wind dynamic pressure is also considered to influence the oscillations of Jupiter's current sheet, particularly in the magnetotail. These questions are difficult to resolve using limited and sparse in situ observations. 

In this work, we use a global magnetohydrodynamics model to study the dynamics of the Jovian magnetosphere. By performing systematic experiments in which we vary the solar wind dynamic pressure, we analyzed the response of the Jovian magnetosphere and tracked the change in the intensity of corotation-enforcement currents, which serve as a proxy for the Jovian aurora. Our model predicts that currents in the ionosphere would decrease after a dynamic pressure enhancement due to an increase in the corotation velocity of the plasma in the dayside magnetosphere. The currents would be more strongly affected on the pre-noon sectors, which correspond to the location of the ‘discontinuity’ seen in the UV auroral observations. 

Using the model, we show that magnetic reconnection occurs on the dayside magnetopause of Jupiter and creates a persistent region of open flux in the polar regions of the planet. Our model also predicts the release of plasmoids due to reconnection in the magnetotail, and we demonstrate that these plasmoids magnetically connect to the polar regions of the planet, which predominantly connect to open field lines in the solar wind. 

We also study the response of the oscillating magnetotail current sheet to changes in the upstream solar wind dynamic pressure. It was previously believed that solar wind driven compression would increase magnetic pressure in the magnetotail lobes, which would inhibit the current sheet oscillations. Using the MHD model, we find that increasing solar wind dynamic pressure also increases the density and temperature in the magnetotail lobes, which reduce the Alfven and magnetosonic speeds in the magnetosphere, which can decrease the wavelength of current sheet oscillations. 

Lastly, our MHD model and other in situ observations have predicted that plasmoids produced in the Jovian magnetotail are large and infrequent. We analyze magnetometer data from the Juno spacecraft and identify magnetic flux ropes and O-lines with durations lasting less than 300 s, which we argue leads to diameters comparable with the ion-inertial length. These findings suggest that despite differences, magnetic reconnection in the Jovian magnetotail operates via current sheet instabilities, similar to that seen at Earth, Mercury and in particle-in-cell simulations. 

