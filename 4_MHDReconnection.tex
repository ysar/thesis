\chapter{Magnetic Reconnection in the Jupiter MHD model}

\section{Introduction}

\section{Reconnection on the dayside for Parker-spiral IMF}

\begin{figure}
    \centering
    \includegraphics[width=0.6\textwidth]{images4/reconnection-dayside.jpg}
    \caption{Caption}
    \label{fig:reconnection-dayside}
\end{figure}

For the Parker‐spiral IMF configuration used in our simulation, reconnection is found to occur primarily on the magnetopause at relatively high latitudes (at $\sim50^\circ$) latitude) where the strongest magnetic shear is present. Figure \ref{fig:reconnection-dayside} shows a snapshot of the simulated magnetopause surface extracted from Run 4. The magnetopause surface is determined by identifying the separatrix between magnetospheric and magnetosheath field lines based on 3‐D field line tracing. The color contours on the magnetopause surface represent plasma flow speeds, and sample field lines are superimposed to show the magnetic topology. As shown, under the spiral IMF configuration with positive By, we find that reconnection takes place mainly in two quadrants in the YZ plane: in the northern hemisphere on the dawnside and in the southern hemisphere on the duskside. The reconnection geometry is consistent with the prediction by the analytical model of \cite{Masters2017} for the same IMF configuration.

\section{Plasmoid release and variation of open magnetic flux}

In all simulations listed in Table 1, tail reconnection occurs and produces plasmoids. For instance, a large plasmoid can be seen in Figure 6, row 3 in the form of a high‐density region between 00 and 06 LT. After initially being created at a radial distance of $\sim$50–70 $R_J$ on the dawnside, the plasmoid is seen to grow and move tailward, eventually escaping the magnetosphere and lost to the solar wind. In Run 4, a vortex structure is created in the magnetosphere on the duskside at around 40 $R_J$ radial distance from the planet (not shown). The vortex is formed subsequent to a large reconnection event in the magnetotail and it strengthens as it moves sunward, eventually reaching the postnoon sector. The vortex is made of corotating and anticorotating flows and produces a strong ionospheric response in the postnoon sector (Figures 8‐4a and 8‐4b near 16 LT). We believe that this vortex and the subsequent localized bright spot in $J_\parallel$ observed near 16 LT in the ionosphere are due to the interaction of return flow from the duskward tail reconnection site with the corotating magnetospheric plasma and has also previously been observed by \cite{Fukazawa2006a} using their MHD model. 

\begin{figure}
    \centering
    \includegraphics[width=0.5\textwidth]{images4/open-flux-variation.jpg}
    \caption{Caption}
    \label{fig:open-flux-hemisphere}
\end{figure}

In Figure \ref{fig:ionosphere-currents} the yellow points superimposed onto the contour plots correspond to the OCB identified in our simulations. For each local time and longitudinal position in the ionosphere, we trace 3‐D magnetic field lines from a sphere at 3 $R_J$ to identify any transition between open and closed field lines. If a transition is found, its location on a 1 $R_J$ sphere is determined by using a dipole field line trace, which is then plotted in Figure \ref{fig:open-flux-hemisphere}. Even with $1^\circ$ resolution in both latitude and longitude, our tracing algorithm does not find any such transition during times when the IMF is southward, which is consistent with the picture that the magnetosphere is largely closed under such external conditions. In contrast, under a Parker spiral IMF, the OCB increases in size with time and can reach a latitude of $\sim80^\circ$ on the nightside under strong solar wind driving (column 4). While the size of the OCB tends to vary depending on the upstream conditions, for the various upstream conditions examined in our simulations it is always located poleward (by at least a few degrees) of the main oval of upward field‐aligned currents arising from corotation breakdown, which lies at $\sim75^\circ$ latitude. 

For further analysis we divide the magnetic field lines extracted from our MHD model into four categories, denoted by the “status” variable (Table 2). A status value of 0 represents a closed field line with both ends connected to the planet. A status value of 1 or 2 implies an open field line with one footprint in the northern or southern hemisphere, respectively, while a status value of 3 refers to those field lines with both ends in the solar wind, which we call disconnected field lines. Figure \ref{fig:open-flux-hemisphere} shows the status of field lines seeded from the northern and southern ionosphere, whereas Figure \ref{fig:status-magnetosphere} shows the status of field lines seeded from the equatorial plane in the magnetosphere. 

\subsection{Magnetic topology associated with plasmoid release}

In Figure \ref{fig:open-flux-hemisphere}, we show the status maps of the northern and southern hemispheres on a 1 $R_J$ sphere, at different times during the sequence of a plasmoid release. For both hemispheres, the cyan regions contain field lines that are closed (status = 0). For the northern hemisphere panels, the dark blue regions contain open field lines (status = 1) that magnetically map to the solar wind. For the southern hemisphere panels, the red regions indicate open field lines (status = 2) that map to the solar wind. It is immediately clear from Figure \ref{fig:open-flux-hemisphere} that these status maps are not north‐south symmetric, with stark differences in the topology between the two hemispheres. 

Two plasmoids are observed in the magnetosphere during the times shown in Figure \ref{fig:open-flux-hemisphere}: a relatively small size plasmoid on the duskside and a much larger plasmoid near dawn. When the plasmoids are initially formed, they contain predominantly closed flux. This is consistent with the idea that plasmoids form due to the Vasyliunas cycle are created on closed field lines. As the plasmoids move tailward, they grow in size and create a region of closed flux inside the polar cap. The large plasmoid in the dawn sector of the magnetosphere can be identified by its status signature on the dawnside in the form of a large region of closed flux, whereas the smaller plasmoid in the dusk sector also creates a similar region of closed flux in the duskward polar cap. With time, the plasmoids grow and interact with the surrounding plasma and magnetic field, which creates rather complicated magnetic field structures that contain intertwined open and closed field lines (Figure \ref{fig:open-flux-hemisphere}b). As the plasmoids move further down the magnetotail, they grow in size and the status signatures associated with plasmoids move toward midnight (previously at dawn and dusk) and the high‐latitude region in the ionosphere starts to be filled with open field lines. With time, the ratio of open field lines to closed field lines in the plasmoid footprint increases in both the northern and southern hemispheres. As a result, the tail plasmoids, when mapped magnetically to the ionosphere, correspond to a stripe‐like structure. 

Observations of the polar aurorae of Jupiter show various intriguing features such as arcs and filaments \cite{Grodent2003a,McComas2007,Nichols2009a} that have been suggested to be linked to dynamic processes in the solar wind and magnetotail. Our simulation results show that the polar regions of the planet, which are often assumed to lie on open field lines, may magnetically connect to distant regions in the magnetotail associated with a plasmoid. While our MHD simulation does not directly model the kinetic physics of particle energization associated with reconnection, the magnetic topology associated with plasmoid release and propagation through the tail region as seen in our simulation suggests that energization associated with tail plasmoid release may provide a plausible explanation for the observed arc‐like or filament‐like aurora structures. 

\begin{figure}
    \centering
    \includegraphics[width=\textwidth]{images4/plasmoid-3d-inset.jpg}
    \caption{Caption}
    \label{fig:plasmoid-3d-inset}
\end{figure}

In Figure \ref{fig:plasmoid-3d-inset}, we show the three‐dimensional magnetic field lines associated with the tail plasmoid along with the plasma density contours in the equatorial plane. Orange field lines are closed field lines, whereas black field lines are “disconnected” field lines with both ends in the solar wind. It can be seen that although the plasmoid is generated on and still contains closed field lines, it is surrounded by open field lines as it moves tailward. The inset in Figure \ref{fig:plasmoid-3d-inset} shows the corresponding ionospheric status map in a similar format as Figure \ref{fig:open-flux-hemisphere}. Since this plasmoid is noticeably smaller, it has a smaller, but consistent, status signature in the form of a region of closed flux in the polar cap on the nightside. 

\subsection{Open flux in the magnetosphere}

\begin{figure}
    \centering
    \includegraphics[height=0.9\textheight]{images4/status-fieldlines-magnetosphere.jpg}
    \caption{Caption}
    \label{fig:status-magnetosphere}
\end{figure}

To complement the analysis of the status of field lines shown in section 6.1, we repeated the same procedure of tracing field lines starting in the equatorial plane of the magnetosphere. The corresponding magnetospheric status maps are shown in Figure \ref{fig:status-magnetosphere} for two different types of plasmoids that we will call Type 1 and Type 2, respectively. The left column shows a plasmoid of Type 1, which is a large plasmoid released on the dawnside, whereas the right column shows a plasmoid of Type 2, which is released near midnight. Both plasmoids have some common features, namely, they both originate from closed field lines. After release, the Type 1 plasmoid severely distorts the magnetic topology of the magnetotail. Upon close examination, one can see regions of closed field lines interspersed within large regions of open field lines. The Type 2 plasmoid, on the other hand, has a cleaner topological fallout. After being detached as a “blob” of closed flux, the Type 2 plasmoid is surrounded by disconnected field lines (status = 3, both ends in the solar wind) even though it is located deep inside the magnetosphere. With time, the Type 2 plasmoid moves tailward and the region of closed flux associated with the plasmoid decreases in size. However, the region of disconnected flux in the magnetotail expands after the release of a Type 2 plasmoid. 

Another feature which can be recognized in Figure \ref{fig:status-magnetosphere} is the stark separation between dayside disconnected field lines and the open (status = 1 and 2) field lines on the dawn and dusk flanks, as can be identified through the vertical demarcation at $x = -40 R_J$ in column 2. We traced 3‐D magnetic field lines which suggest that this vertical demarcation is linked to the draping of the IMF around the magnetopause. That field lines in the magnetosheath drape around the magnetopause has been discussed in detail for Earth and Saturn \cite{Crooker1985,Sulaiman2014,Sulaiman2017Large-scaleMagnetosphere} and is expected to be more pronounced at Jupiter due to the large polar flattening of the magnetosphere \cite{Erkaev1996,Farrugia1998,Slavin1985}. While our model does predict the draping of the IMF around Jupiter's magnetopause, the degree of polar flattening in our model is lower than previous predictions ($\epsilon=\sim$0.3, expected to be $\sim$0.8 according to \citeA{Slavin1985}). 

\subsection{Rate of change of open flux in the magnetosphere}

\begin{figure}
    \centering
    \includegraphics[width=0.75\textwidth]{images4/open-flux-time.jpg}
    \caption{Caption}
    \label{fig:open-flux-time}
\end{figure}

After identifying the status of each point on the 1 $R_J$ sphere for multiple times in our simulations, we integrate the open magnetic flux within the open field region in the northern hemisphere of the planet. Figure \ref{fig:open-flux-time}a shows the variation of this calculated open flux in our model as a function of simulation time for Parker‐spiral IMF (purely $B_Y$) but different solar wind dynamic pressures. The black points show the open flux calculated in our simulation, while the dashed red vertical line marks the time when the introduced forward shock arrives at the bow shock. To reveal potential correlation between plasmoid release and open flux variations, we overlay solid lines in this figure to mark the times when plasmoid release occurs in the simulation. We identify plasmoids in the model based mainly on the $B_Z$ component (the normal component to the tail current sheet). A bipolar variation of $B_Z$ in the equatorial plane is an indication that a reconnection event has occurred in the magnetotail. Typically, plasmoids generated in our model tend to grow in size as they move tailward. Therefore, we further divide the identified plasmoids into two groups based on their maximum size in the cross‐tail direction ($Y$ direction): large plasmoids which have a cross‐tail width larger than 50 $R_J$ at their maximum extent and small plasmoids whose maximum width is $<50 R_J$. Green thick lines and thin blue lines represent the times when large and small plasmoids are released, respectively.

Prior to the shock arrival at $t = 302$ hr, the IMF along with the solar wind parameters remain fixed. During this interval, the open flux in our model gradually builds up due to the magnetopause reconnection. At around $t = 223$ hr (marked by the solid green vertical line), a relatively large plasmoid with a cross‐tail width exceeding 50 $R_J$ forms in the magnetotail that closes some of the open flux stored in the tail lobes, which can be seen as the change of slope in the time history of the open flux. During this period, there are also a couple of smaller‐scale plasmoids (with cross‐tail width $<50 R_J$) formed, as marked by the solid blue vertical lines in Figure \ref{fig:open-flux-hemisphere}. After the shock arrival at $t = 302$ hr, the rate at which the open flux is added to the polar cap increases due to the enhanced solar wind convectional electric field associated with the shock. About 25 hr after the shock impact, a large‐size plasmoid is formed and released in the tail that results in a significant reduction of the open flux. After the impingement of the shock, the compressed magnetosphere experiences frequent plasmoid release, both large and small. Compared to the situation seen in the simulation during the nominal solar wind conditions where plasmoid release occurs every 20 to 50 hr, the occurrence rate is significantly higher in the compressed case, which is of the order of one plasmoid every few hours. A similar behavior has been seen in the MHD model of Saturn by \cite{Jia2012} who found more frequent plasmoid releases during periods of stronger solar wind driving. 

The time variation of the open flux provides a useful measure of how the magnetosphere responds globally to the solar wind driving and internal dynamics. As discussed above, dayside reconnection would add open flux to the polar cap whereas tail reconnection would potentially close open flux stored in the tail lobes. Therefore, the time rate of change of the open flux can be used to quantify the global reconnection efficiency, which depends on the difference in the reconnection rates between the dayside magnetopause reconnection and the tail reconnection. At the beginning of the simulation, in the absence of tail reconnection, we find that the open flux increases at a rate of $\sim$284 kV, which corresponds approximately to the global reconnection rate under the solar wind conditions listed in Table 1, column 2.

In Figure \ref{fig:open-flux-time}b, we show the calculated rate of change of open flux in the northern hemisphere (status = 1), that is, $d\Phi/dt$ as a function of simulation time. After the shock is introduced in the simulation, the rate of increase of open flux increases, corresponding to a peak global reconnection potential of $\sim$2 MV. This increase in the reconnection rate on the dayside is primarily due to enhanced solar wind speed and increased IMF strength due to compression and hence the convectional electric field behind the shock. At later times, the open flux in our simulation is found to decrease and increase periodically at a period of $\sim$20 hr, highlighting the competing influence of magnetopause reconnection (which serves to open magnetic flux) and nightside reconnection (which decreases the net open magnetic flux). Closer examination reveals that the decreases in open flux are also correlated with the release of large plasmoids. \citeA{Walker2016} report on simulations of the Jovian magnetosphere performed by \citeA{Fukazawa2006a} and also found quasi‐periodic increase and decrease in open flux with a similar period of $\sim$20–30 hr. 

In discussing Figure \ref{fig:open-flux-hemisphere} we noted that the release of plasmoids creates a region of open flux in the polar cap, which may seem contradictory to these findings. However, it must also be noted that the overall size of the polar cap also depends on many other factors, such as the difference between reconnection rate on the dayside versus the nightside. Figure \ref{fig:status-magnetosphere} clearly demonstrates that plasmoid release increases the amount of disconnected flux in the magnetosphere. Since the disconnected field lines, by definition, cannot magnetically map to the northern hemisphere, they are not accounted for in our calculation for net open flux which is done on a 1 $R_J$ sphere for Jupiter (thereby only considering status = 1 type field lines). Figure \ref{fig:status-magnetosphere} also shows that with the increase of disconnected flux in the magnetotail, the amount of connected open flux (i.e., status = 1 and 2) decreases. This would decrease the overall size of the polar cap, which would lead to decreased status = 1 flux. The overall shrinking of the polar cap can also be seen in Figure \ref{fig:open-flux-hemisphere}. 

As time progresses the dayside and nightside reconnection rates seem to approach steady state, which can be seen in Figure \ref{fig:open-flux-time}b where fluctuations in $d\Phi/dt$ decrease with time. For the compressed magnetosphere, at the end of our simulation ($t = 400$ hr) the total open flux amounts to $\sim$ 120 GWb. It is interesting to note that the creation of open flux is largely due to the reconnection on the magnetopause, and the result that the net open flux seems to reach a steady state implies that flux closure on the nightside or elsewhere is happening in a manner expected by the terrestrial‐like Dungey cycle. Although we have not yet identified any preferential spatial location where flux closure is consistently occurring, it is clear that both Vasyliunas cycle reconnection (detachment of plasmoids on closed field lines) and Dungey cycle‐type flux closure contribute to the circulation of magnetic flux in Jupiter's magnetosphere. 

Plasmoids generated in Jupiter's magnetotail may be a result of a near‐planet like flux closure event attributed to the Dungey cycle or a result of centrifugal stresses exerted on the corotating plasma, that is, the Vasyliunas cycle, both of which may cause reconnection onset on closed field lines. When the IMF is southward (Run 1), absence of dayside magnetopause reconnection would essentially shut off the Dungey cycle. However, plasmoids are still observed in this case (not shown), and they are a direct product of the Vasyliunas cycle. In this case the plasmoid, once generated, is constrained by the surrounding closed field lines, and “escapes” through the magnetopause. In contrast, when the IMF is in the Parker‐spiral configuration, dayside magnetopause reconnection would add open field to the tail lobes. In this scenario, plasmoids generated due to a tail reconnection event may induce closure of open flux in the tail lobes \cite{Cowley2008} regardless of the original cause of reconnection onset. The lobe reconnection‐produced field lines, which are carried by fast‐moving reconnection jets moving behind the plasmoids, would facilitate the escape of plasmoids down tail. These findings from our Jupiter simulations are similar to those reported for global simulations of Saturn's magnetosphere \cite{Jia2012}. 

As noted earlier, the global simulation presented here is based on an ideal MHD model, in which no kinetic physics is included to describe reconnection. However, reconnection does occur in MHD simulations, which is facilitated by numerical resistivity. It is interesting to compare the global reconnection rate and the resultant amount of open flux in our MHD model with prior estimates based on observations and analytical models. For instance, \citeA{Masters2017} presented an analytical method to estimate the total reconnection potential at Jupiter's magnetopause under different solar wind conditions, and he predicted a dayside reconnection potential ranging between 200 and 1,000 kV. The reconnection potentials estimated in our simulations are in general agreement with the Masters model results. Further, based on auroral observations and magnetic field modeling, \citeA{Nichols2006Magnetopause5AU,Vogt2011a} estimated the typical amount of open flux present in Jupiter's magnetosphere, and their results give a range of 300–700 GWb. The maximum amount of open flux seen in our simulations is about 175 GWb, which is slightly lower than previous estimates and could be related to our use of an ideal axisymmetric dipole for the planetary magnetic field.