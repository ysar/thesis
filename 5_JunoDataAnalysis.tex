\chapter{Reconnection in Jupiter's magnetosphere: Juno observations at the ion-inertial scale}

\section{Introduction}

Simultaneous magnetic reconnection at multiple x-lines can produce magnetic field lines with a helical topology \cite{Hones1984StructureActivity, Slavin2003GeotailSheet}. At scales comparable to the ion-diffusion region, this multiple x-line reconnection has been hypothesized to be the result of a tearing instability acting on a current sheet which has thinned to length scales comparable to the ion-diffusion region \cite{Drake2006ElectronReconnection, Daughton2011RolePlasmas, Lapenta2015SecondaryFronts}. These helical structures are frequently observed in the magnetotails of magnetized planets, and are referred to as ``plasmoids'' or ``o-lines''. Many such plasmoids contain strong plasma pressure gradients, which balance the magnetic forces in the helical structure \cite{Hones1984StructureActivity, Slavin1989CDAWAssessment, Kivelson1995ModelsPlasmas}. However, some structures, called magnetic ``flux-ropes'' do not possess appreciable plasma pressure, and instead contain a strong core field \cite{Sibeck1984MagnetotailRopes, Moldwin1991PlasmoidsRopes} which often exceeds the magnetic field strength in the vicinity of the event and is balanced by the magnetic tension force due to the outer helical wraps \cite{Kivelson1995ModelsPlasmas, Slavin1995ISEETopologies}. If the magnetic force on the structure is negligible, i.e. $\mathbf{J}\times\mathbf{B} = \nabla p = 0$, then these flux ropes are termed to be \emph{force-free}. These quasi-force free flux ropes correspond to the minimum energy state for a flux rope that all such structures will evolve toward with increasing time \cite{Taylor1974RelaxationFields, Priest2013TheLecture}.

Comprehensive investigation in the terrestrial magnetotail, courtesy of multiple spacecraft spread over many years, together with increasingly complex numerical models \cite{Drake2006ElectronReconnection, Drake2006FormationReconnection} have shown that flux ropes in the the magnetotail are quite diverse in terms of size - with diameters ranging from thousands of km to below the ion inertial length, which is typically on the order of $\sim$hundreds of km \cite{Eastwood2016Ion-scaleMMS, Sun2019MMSSheet}. A similar dichotomy in flux rope diameter is seen at Mercury, whose magnetosphere is similar to that the Earth in so-far that it is driven by the Dungey cycle. Flux ropes in Mercury's magnetotail have been observed to have diameters of hundreds of km \cite{DiBraccio2015MESSENGERMagnetotail, Slavin2009MESSENGERMagnetosphere}. However, large plasmoids with diameters comparable to Mercury's radius have also been observed \cite{Zhong2019Magnetotail}.

Magnetic reconnection is also known to occur in the magnetospheres of the gas giants, Jupiter and Saturn, and plasmoids have also been observed in these systems primarily by the Galileo and Cassini spacecraft \cite{Russell2000SubstormsTail, Woch2002a, Kronberg2007AMagnetosphere, Kronberg2008MassParameters, Vogt2010a, Vogt2014, Jackman2011CassiniSaturn}. Due to fast rotation and the strong internal magnetic field, the Dungey-cycle is considered to play a minor role at Jupiter. In this case, reconnection in the magnetotail is initiated by the Vasyliunas cycle \cite{Vasyliunas1983a}. In this process, mass loaded flux tubes stretch freely on the nightside and lead to thinning of the magnetotail current sheet, thereby detaching plasmoids in the process \cite{Kivelson2005DynamicalMagnetosphere,Cowley2015Down-tailMagnetospheres}. These plasmoids also been observed in numerical models of Jupiter's \cite{Fukazawa2010a,Chane2017a,Sarkango2019} and Saturn's magnetosphere \cite{Jia2012}. However, in-situ observations are limited to observations by single-spacecraft, which lack global coverage. On the other hand, numerical models are limited by the fluid approximation and other aspects such as coarse grid spacing in region far from the planet.

On the basis of in-situ observations, it was estimated that Jovian plasmoids have a diameter between $\sim2$ to 20 R$_J$, with plausible cross-tail widths of $40$ to $70$ R$_J$ \cite{Vogt2014}. These numbers could only account for $30$ to $210$ kg/s plasma loss due to magnetotail reconnection, which is substantially less than the plasma source due to ionization of Iogenic neutrals, estimated to be between $250$ to $1000$ kg/s. To explain this discrepancy, \citeA{Cowley2015Down-tailMagnetospheres} argue that the size of the plasmoid event is likely underestimated, as the plasmoid encompasses a long time interval and cannot be described accurately as having a circular cross-section. Other mechanisms - such as the recurrent release of small plasmoids \cite{Kivelson2005DynamicalMagnetosphere} or a diffusive ``drizzle'' across weak equatorial field lines in the magnetotail \cite{Bagenal2007ThePoles} have also been put forward. 

% ---- copied text -----
Plasmoids and flux ropes observed so far in the Jovian magnetosphere have been fairly large. \citeA{Vogt2014} report that most such Jovian events lack a core field and are magnetic O-lines, though they discuss a few observations of magnetic flux ropes as well. The mean duration of the observed plasmoids and flux ropes observed by the Galileo spacecraft at Jupiter was determined by \citeA{Vogt2014} to be 6.8 minutes and by \citeA{Kronberg2008MassParameters} to be between 10 and 20 minutes (The two studies use different definitions for the duration of a plasmoid event). \citeA{Vogt2014} estimated the average diameter of the plasmoid to be approximately 2.6 R$_J$ (where 1 R$_J$ = 71492 km) or 1.85 $\times$ 105 km, though they note that because of single-point measurement limitations, these plasmoid sizes could be larger. Assuming that the equatorial plasma density at a distance of 90 R$_J$ down-tail is $\sim$0.01 cm$^{-3}$ \cite{Bagenal2011b} and that the plasma is made up of mostly S$^+$, S$^{++}$, O$^+$, and H$^+$ ions \cite{Kim2020SurveyObservations}, we can approximate a mass of 16.6 $m_p$ for the average ion (where $m_p$ is the mass of a proton) and estimate an ion inertial length ($d_i=c/\omega_{pi}$) of approximately 104 km, which is at least an order of magnitude smaller than the diameter of the plasmoids seen by Galileo. Considering that the Galileo magnetometer had a cadence of a few seconds per vector, it would have been impossible to detect sub-ion scale flux ropes or O-lines, whose in-situ signatures would last only a few seconds. 

The dichotomy seen at the other planets and in simulations of reconnecting fields leads to a natural question of whether ion-scale flux ropes exist in the Jovian magnetotail and if they can be identified using the high-resolution capabilities of the Juno instrument suite. Recent plasmoid observations by the Juno spacecraft reported by \citeA{Vogt2020MagnetotailObservations} have corroborated the Galileo observations, in that large plasmoids lasting several minutes on average were observed. In this work, we extend upon previous Galileo and Juno investigations and present two ion-inertial scale flux ropes observed by Juno in the dawn-side Jovian magnetotail, which lasted roughly 22 seconds and 62 seconds.  A force-free flux rope model is used to fit the Juno magnetic field observations, and it is determined that these flux ropes were quasi-force free, indicating that the thermal pressure plays a minor role compared to magnetic pressure. Estimates of the local plasma density surrounding these flux ropes are made using the low-frequency cutoff for the continuum radiation as observed by the Juno Waves instrument (e.g. \citeA{Barnhart2009ElectronSpectra}), which is then used to infer the ion inertial length. This study is the first reported observation of magnetic flux ropes on the ion scale in Jupiter’s magnetosphere. These observations show that while reconnection on the global scale at Jupiter’s magnetosphere is likely influenced by the Vasyliunas cycle, as evidenced by the large plasmoids seen by both Galileo and Juno; small-scale reconnection and secondary magnetic islands generated in the Jovian magnetotail are likely due to current-sheet instabilities, similar to observations at Earth and Mercury.


\section{Methodology}
We use high-resolution magnetometer data in the Jupiter De-Spun Sun (JSS) coordinate system.  The Z axis for the JSS system is aligned with Jupiter’s north pole, X points towards the sun and Y completes the right handed coordinate system. Also used are the corresponding magnetic field components in the spherical polar JSS system ($B_r$, $B_\theta$, $B_\phi$) referring to the radial, co-latitudinal and azimuthal directions.

The Juno Magnetometer investigation measures the magnetic field strength and direction ambient to the spacecraft using boom-mounted fluxgate magnetometers \cite{Connerney2017TheInvestigation}. The magnetometer measures at rates of 16 to 64 vectors/second. These high cadence rates are significantly greater than what was returned by the Galileo magnetometer (between 24 s to 60 s per vector, see e.g. \citeA{Vogt2010a,Vogt2020MagnetotailObservations}) and they allow us to study smaller scale structures durations down to $\sim$100 msec. We also use data from the Juno Waves instrument \cite{Kurth2017TheInvestigation}, which measures the fluctuations in the electric field between 50 Hz and 40 MHz and in the magnetic field from 50 Hz and 20 kHz. We use the low frequency cutoff for the continuum radiation to infer the electron density \cite{Barnhart2009ElectronSpectra}. 

Juno orbits Jupiter in a highly elliptical trajectory, with each perijove pass separated by $\sim$53 days. At Jupiter orbit insertion, the apogee was located at $\sim$06 local time (LT), but over time has precessed into near midnight regions, dropping from $\sim$0 R$_J$ to roughly 40 R$_J$ in the negative $Z_{JSS}$ direction as of early 2019. It can be seen from Figure 1 that the orbits spend a reasonable amount of time in the equatorial / low latitude magnetotail region, which enabled Juno to capture multiple current sheet crossings almost every perijove pass.

In this study, as in \cite{Vogt2010a,Vogt2014}, positive values of B$_\theta$ indicate a field pointing in the negative $Z_{JSS}$ direction at the equator. In the quiet state with Jupiter’s magnetic moment pointing north, the equatorial magnetic field is primarily in the positive $\theta$ (negative Z, assuming no current sheet tilt) direction. The magnetic signature of a tailward-moving plasmoid or flux rope passing over a spacecraft near the equatorial plane is primarily observed in the B$_\theta$ component as a slight increase and subsequent reversal to negative values (see e.g. Figure 2 of \citeA{Vogt2014}). As the plasmoid or flux rope passes over the spacecraft, the return to positive values can either be symmetric, hinting at reconnection occurring in closed field lines, or gradual, indicative of a post-plasmoid plasma sheet that is formed when reconnection has progressed to the tail lobes \cite{Jackman2011CassiniSaturn,Jia2012}. Conversely, planetward moving loop-like or helical magnetic structure would exhibit the opposite signature i.e. an increase of B$_\theta$ in the negative direction and a reversal to positive values. If the plasmoid possesses a core field, it should typically be identified by a peak in the cross tail component, either B$_y$ or B$_\theta$ as well as a corresponding peak in the magnetic field strength which roughly matches the time where the reversal in B$_\theta$ is observed. Most magnetic structures observed in Jupiter’s plasma sheet (e.g. \citeA{Vogt2014,Vogt2020MagnetotailObservations}) lack an axial core field and are identified as plasmoids or O-lines. This result is similar to what has been observed at Saturn \cite{Jackman2011CassiniSaturn}. The reason could be due to large plasma pressure in a high $\beta$ plasma and their primary role of carrying plasma away from these planets and balancing the plasma derived from their moons \cite{Kivelson1995ModelsPlasmas,Cowley2015Down-tailMagnetospheres}.

Using the high-resolution Juno data, we searched (by eye) for bipolar variations in the B$_\theta$ component in proximity to current sheet crossings to identify possible plasmoid and flux rope signatures which are roughly one minute or less in duration. To verify whether each event was a plasmoid or a magnetic flux rope, we performed the minimum variance analysis (explained below) to infer the orientation of the structure.

The minimum variance analysis (MVA) can be used to identify the orientation of a flux rope with respect to the magnetotail (e.g. \cite{Sonnerup1967MagnetopauseObservations}). The eigenvectors of the covariance matrix corresponding to the three eigenvalues (in increasing magnitude) $\lambda_1$, $\lambda_2$, $\lambda_3$, represent the directions of minimum, intermediate and maximum variance, respectively. For magnetic flux ropes, which possess a helical field on the outside and a unidirectional axial field on the inside, the axial direction can be inferred using the eigenvector of intermediate variance ($\mathbf{x}_2$). There are additional criteria required to identify a flux rope using MVA: A bipolar signature in the maximum ($B_3$) varying component should be present and the eigenvector of the maximum variance should be predominantly in the direction normal to the current sheet. The ratio of maximum to intermediate ($\lambda_3/\lambda_2$) and intermediate to minimum ($\lambda_2/\lambda_1)$ eigenvalues must be relatively large (ideally larger than 3 or 4, e.g. \citeA{Lepping1990MagneticAU}) for the orthogonal coordinate system to be well-defined. A rotation should be observed in the $B_2$-$B_3$ hodogram. An almost zero $B_1$ indicates that the spacecraft passed close to the center of the flux rope or O-line. For ta flux rope, the core field should be seen as an enhancement in the $B_2$ component, whereas for a plasmoid, a local minimum in the $B_2$ component would be seen.

Following the procedure of \cite{Lepping1990MagneticAU}, we also fit a constant alpha force free flux rope to selected events. Under the force-free assumption, pressure gradients and the $\mathbf{J}\times\mathbf{B}$ force are considered to be negligible. In this case, the Lundquist solutions can be used to model a circular force-free flux rope \cite{Lepping1990MagneticAU,Slavin2003GeotailSheet} as, 
\begin{align}
    B_A & = B_0 J_0 (\alpha r) \\
    B_T & = B_0 H J_1 (\alpha r)
\end{align}

Where $B_A$ and $B_T$ are the axial and tangential field components, $B_0$ is the core field strength, $\alpha$ is a constant parameter, $r$ is the distance to the center of the flux rope normalized to the radius of the flux rope and $J_0$ and $J_1$ are Bessel functions of the first kind. Since the velocity at which the plasmoid moves is unknown, we cannot estimate the radius of the flux rope directly. Instead, we estimate the Impact Parameter (IP), defined as the ratio of distance to the center at closest approach to the radius of the flux rope. The parameters to fit are the core field strength ($B_0$), Impact Parameter (IP), orientation of the flux rope axis ($\theta_A$,$\phi_A$) and the handedness $H$. These parameters are varied so as to minimize the reduced $\chi^2$ defined as \cite{Lepping1990MagneticAU},

\begin{equation}
    \chi^2_r = \frac{1}{N} \sum_i \left[ \left( B_x - B_{x,m}\right)^2_i + \left( B_y - B_{y,m}\right)^2_i + \left( B_z - B_{z,m}\right)^2_i \right] 
\end{equation}

Where $B_m$ is the modeled field and $N$ is the number of data points in the reconstruction. The intermediate direction provided by MVA is used as the initial guess for the orientation of the flux rope and the minimization is done in two steps, first with a unit normalized magnetic field with $B_0=1$ nT and then with $B_0$ as a variable parameter to fit the core field strength. The complete force-free fitting algorithm can be found in Appendix A of \cite{Akhavan-Tafti2018MMSMagnetopause}, with the only difference being that in our study the length scales are normalized to the radius of the flux rope, which is unknown.

\section{Discussion}
\subsection{Event 1 - Flux rope (DOY 234, 2017)}

On DOY 236, 2017 Juno was located 74.3 $R_J$ away from Jupiter at approximately 04 LT (dawnside magnetotail) when it encountered a flux rope between 20:21:15 and 20:21:37 UTC. The sign of $B_\theta$ was positive before and after this event, but briefly reversed to negative values during the interval (Figure 2a-2d). The positive $B_\theta$ before and after the bipolar signature is consistent with Juno being in the near-Jupiter plasma sheet where the inward magnetic stress exerted by the stretched, closed magnetic field is balanced by the inward gradient in the plasma pressure.  $B_r$ is less than 1 nT during the encounter and $B_\phi$ increases (in the negative) by approximately 2 nT, which is the core field of the flux rope. The difference between the extrema in $B_\theta$ is about 4 nT. The sharp peak in the magnetic field strength, closely aligned with the center of the $B_\theta$ reversal, is a characteristic signature of a flux rope. The flux rope is close to the current sheet, as evidenced by the reversal of $B_r$ from positive to negative values before and after the event. Although there is both a positive-to-negative and negative-to-positive polarity reversal of $B_\theta$, the core field peak is seen during the negative-to-positive reversal, which hints that the flux rope was traveling planetward.

After performing the MVA, we find a bipolar variation in the $B_3$ (maximum) component and a peak in the $B_2$ (Figure 2f-2h), which is expected for a flux rope with a core field. The ratio between the intermediate and minimum eigenvalues of the variance matrix is 4.7, whereas the ratio between the maximum and intermediate values is 28.76. Looking at the $B_2$-$B_3$ hodograms shown in Figure 2 (i) and (j), we can observe a rotation of the magnetic field. Figure 2 also shows the magnetic field components of the modeled force-free flux rope (in blue) in the MVA coordinate system which best fits the data (minimum $\chi_r^2$=0.13). The modeled flux rope has a core field strength of 3.86 nT and an impact parameter of 0.0, which indicates that the spacecraft passed very close to the center of the flux rope structure. This is also supported by the extremely low magnitude of $B_1$ (less than 0.4 nT). 

The eigenvectors of the variance matrix in the direction of minimum, intermediate and maximum variance are (in the cartesian JSS coordinate system), $\mathbf{x}_1 =(-0.03,0.86,-0.5)$, $\mathbf{x}_2=(-0.98,0.12,-0.14)$ and $\mathbf{x}_3=(-0.18,-0.49,0.85)$. Although flux ropes in the terrestrial magnetotail typically have a core field in the Y direction (as provided by $\mathbf{x}_2$), we find that for this event the direction of intermediate variance is in the X direction, which is close to azimuthal direction at the given spacecraft location (Figure 1). 

In total the peak-to-peak duration for this event was 22 seconds. The cutoff for the continuum radiation was roughly between 500 and 600 Hz as observed by the Waves instrument (Figure 2e), though there is a narrowband interference feature at 500 Hz lasting the entire duration shown in Figure 2e which makes identification of the cutoff ambiguous below this frequency. Using this cutoff, which is expected to occur at the local electron plasma frequency \cite{Barnhart2009ElectronSpectra}, we can infer the plasma density to be approximately 0.003 cm$^{-3}$, which corresponds to an ion inertial length (assuming $m_i$=16 amu) of 16660 km (or  0.23 $R_J$). The background magnetic field surrounding this event is approximately 5 nT, which leads to an Alfven speed ($v_A=B/\sqrt{\mu_0 \rho}$) of 480.6 km/s in the lobes. \cite{Kronberg2008MassParameters} found that most energetic particle bursts corresponding to plasmoid events have speeds of roughly 450 km/s. These two speeds lead to estimates of the flux rope diameter to be between 9900 km and 10574 km (0.14 to 0.15 $R_J$),  which is comparable to the ion diffusion region.

After the flux rope has passed over the spacecraft, a reversal in the guide field ($B_\phi$) is observed from -4 nT to 2 nT. This reversal of the out-of-plane component of the magnetic field in close proximity to the reconnection x-line could be due to the quadrupolar Hall magnetic field (Sonnerup 1979; \cite{Eastwood2007Multi-pointOn}, which is formed due to the decoupling of ions and electrons in the ion diffusion region and has been identified by multiple spacecraft in the terrestrial magnetotail \cite{Nagai2001GeotailMagnetotail}. We caution however that single-spacecraft measurements are unreliable to conclusively determine whether or not the reversal in $B_\phi$ is due to the Hall field. Another possible explanation for the reversal could be related to the bend-back of the magnetic field, which has been seen as a correlation between the sign of $B_r$ and $B_\phi$. In the present situation, the latter theory is less likely since $B_\phi$ returns to negative values despite multiple current sheet crossings as seen in $B_r$. 

\subsection{Event 2 - Flux rope (DOY 338, 2017)}

On DOY 338, 2017 between 01:49:57 and 01:50:59 UTC Juno was located at 92 $R_J$ between 03-04 LT and observed a reversal in $B_\theta$ from positive to negative values, indicating a tailward moving flux rope (Figure 3a-3d). Unlike the previous example, the magnetic field magnitude did not peak inside the event interval. The azimuthal field component remained close to zero, whereas the radial component peaked in the middle of the event interval. 

Performing the MVA provides us with additional information (Figure 3f-3h) – the maximum variance is in the Z direction ($\mathbf{x}_3=(-0.07,0.01,1.00)$), as expected, whereas the intermediate and minimum variance directions lie in the XZ plane close to the local radial and tangential directions. The component of the magnetic field in the minimum variance direction is close to zero. The intermediate component ($B_2$) peaks in the middle of the event interval. The $B_2$-$B_3$ hodograms show a clear rotation of the magnetic field. However, $|\mathbf{B}|$ does not peak at the center of the interval and the best fit force-free flux rope does not fit the data well ($\chi^2_r=5.9$), although the modeled field in the $B_2$ component looks reasonable. While conventionally flux ropes in the terrestrial magnetotail are seen to possess a strong core field, this has not been the case for the giant planet magnetospheres. Plasmoids observed at Jupiter and Saturn usually lack a core field, which is likely due to the presence of plasma with a large plasma $\beta$. The force-free model is based on the assumption that pressure gradients inside and surrounding the flux rope are negligible, which may not be the case for Jupiter and Saturn. Another possible explanation is that this is a flux rope in the early stages of formation and has not yet reached the minimum energy force-free state. 

This event lasts 62 seconds between the two $B_\theta$ peaks. The cutoff for the radio emission was measured to be $\sim$1 kHz (Figure 2e), which corresponds to an electron density of 0.012 cm$^{-3}$ or an ion inertial length of ~8330 km (0.11 $R_J$), assuming $m_i=16$ amu. The background lobe magnetic field during the event was approximately 4.5 nT, which leads to an Alfven speed of 216 km/s. If we assume that the speed of the flux rope is limited by the Alfven speed in the low-latitude magnetotail lobes, the flux rope diameter would be approximately 13410 km (0.19 $R_J$), $\sim$1.6 times larger than the ion inertial length. Note that the plasma density as seen by Waves for Event 2 is a factor of $\sim$4 larger than that for Event 1, even though the second event is observed at a radial distance of 92 $R_J$ as opposed to the first, seen at 75 $R_J$.

The MVA analysis shows that Juno is sampling the portion of the flux rope where the axis is almost radial, as determined by the direction of intermediate variance. The ratio of the maximum to intermediate and intermediate to minimum eigenvalues are quite large ($\lambda_3/\lambda_2=7.97$,$\lambda_2/\lambda_1=81.39$), meaning that the coordinate system is well defined. Note that observations of flux ropes in the terrestrial magnetotail have shown that many flux ropes are tilted in the plane of the current sheet \cite{Slavin2003GeotailSheet}.

The spectra for the electric field as observed by the Waves instrument for Event \#2 is shown in Figure 3e. A broadband intensification can be seen between 1-3 kHz for the duration of this event. Enhanced fluctuations in the electromagnetic field have been seen inside plasmoid intervals in the past in the terrestrial magnetosphere \cite{Kennel1986PlasmaRopes}. Although the continuum radiation is observed during the first event as well, no transient intensification was observed due to the flux rope.