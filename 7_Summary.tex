\chapter{Summary}

In this dissertation, we have used a combination of numerical modeling and  data analysis to study the dynamics of the Jovian magnetosphere. The following sections summarize our findings.

\section{Summary}
We used a global magnetohydrodynamics model to simulate the Jovian magnetosphere and its interaction with the upstream solar wind as well as with the Jovian ionosphere. 

In Chapter 3, we studied the response of solar wind dynamic pressure enhancements on corotation and corotation-enforcement currents in the ionosphere, which has been suggested to link closely to Jupiter's main aurora. Through systematic numerical experiments, we found that an increase in solar wind dynamic pressure actually decreases the strength of field-aligned currents on the dayside, with minor enhancement on the nightside. By analyzing the magnetospheric response, we proposed that this reduction can be explained by the increase in corotation speed of the magnetospheric plasma near the sub-solar magnetopause. The increase in plasma azimuthal velocity had been proposed by earlier theoretical models to be a consequence of angular-momentum conservation due to the solar wind driven compression of the corotating plasma. Another possible explanation for the increase in azimuthal velocity is due to returning flows from the tail reconnection site, which merge with and accelerate the magnetospheric plasma on the dawnside. Our results contradict existing opinions about the positive correlation between solar wind dynamic pressure and the brightness of the Jovian aurora.

In Chapter 4, we investigated the role of magnetic reconnection in the Jovian magnetosphere. Unlike the terrestrial magnetosphere, the Jovian magnetosphere was assumed to be insensitive to changes in the solar wind and interplanetary magnetic field. It was believed that magnetic reconnection on the dayside magnetopause, even if it did occur, would not form persistent open field lines, which map to the polar regions of the planet. At the same time, in situ observations have shown that reconnection does occur in the Jovian magnetotail. Moreover, the polar UV aurorae of Jupiter have been observed to be highly dynamic, varying on much shorter time-scales than the main auroral oval.

Using our MHD model, we found that reconnection does occur on the dayside magnetopause at Jupiter, and thus creates open field lines in the polar cap. By tracing magnetic field lines in the magnetosphere, we studied the topology of plasmoids, which are products of magnetotail reconnection. We discovered that plasmoids were created due to internal processes initially within closed field lines. As time progressed, the plasmoids grew in size and escaped the magnetosphere via the magnetotail. By mapping the plasmoids to the ionosphere, we found that plasmoid release creates a region of closed flux inside the previously open polar cap, which gradually refills with open flux as the plasmoid travels to the distant magnetotail. Our results support the hypothesis that the dynamic polar aurora of Jupiter can be related to distant processes in the magnetotail, while at the same time, remaining open to the solar wind. 

In Chapter 5, we investigated the role of solar wind dynamic pressure on the morphology of the Jovian current sheet. The magnetotail current sheet at Jupiter oscillates at the planetary rotation period due to the tilt between the magnetic dipole and the rotation axis. It was presumed and supported by empirical models that these magnetotail oscillations were inhibited at distances far from the planet due to the presence of the magnetotail; ultimately a result of the solar wind's interaction with the internal field. By incorporating a more realistic internal field model into our MHD model that includes the $10^\circ$ dipole tilt, we studied the behaviour of the current sheet under different solar wind dynamic pressures. 

We found that increasing solar wind dynamic pressure, rather than inhibiting the oscillations of the current sheet, actually increase its \emph{wavy}-ness by reducing the wavelength of these oscillations. The wavelength is intricately connected to the speed at which these current sheet waves propagate from the planet. Previous models had assumed that an increase in solar wind dynamic pressure would increase the magnetic pressure in the magnetotail lobes, which would inhibit the current sheet oscillations. However, by analyzing our simulations, we showed that an increase in solar wind dynamic also increases plasma density and temperature in the magnetotail. Collectively, these parameters serve to reduce the Alfven and magnetosonic speeds in the magnetotail lobes, contrary to intuition.

In Chapter 6, we study in situ magnetic field and energetic plasma data from the Juno spacecraft to form a holistic picture of reconnection in the Jovian magnetotail. Previous observations and MHD simulations had predicted that plasmoids created via tail reconnection at Jupiter were large and infrequent (with timescales of several days). By using the high-resolution magnetometer data from the Juno spacecraft, we identified several candidates for plasmoids whose in situ signatures last for $\sim1$ minute. By using the local Alfven speed and density, we estimated that these durations correspond to plasmoid diameters comparable to the ion inertial length, assuming the dominance of heavy ions such as oxygen and sulfur. The presence of ion inertial scale plasmoids in the Jovian magnetotail suggests that magnetic reconnection proceeds there via current sheet instabilities, similar to observations at Earth and Mercury, despite differences in the initial mechanisms which can cause the current sheet to thin. 

\section{Future work}

Our study leads to additional questions which can be future directions for study - 

\begin{enumerate}
    \item The physical mechanism which leads to the brightening of the Jovian aurora roughly 1 rotation period after the solar wind compression remains a mystery. The discrepancy between numerical models and observations of the Jovian main aurora cannot be answered conclusively without a solar wind monitor and multi-spacecraft or high-fidelity propagation of the solar wind from 1 AU to Jupiter's orbit.
    
    \item Recent work has criticized the corotation enforcement current system theory \cite{Bonfond2020SixJupiter} regarding the formation of the Jovian aurora, yet most MHD models create this current system self-consistently. The physical origin of Jupiter's UV aurora remains a mystery and needs to be investigated on a more fundamental level. Is the aurora produced due to discrete acceleration and precipitation of electrons, or are wave-particle interactions more important?
    
    \item The observations of small-scale reconnection can have large implications on mass loss due to plasmoid release. How frequently does magnetic reconnection occur and what is the contribution of small plasmoids? How does reconnection occur on the dusk-side of the magnetotail?
    
    \item Does Jupiter's magnetosphere contain open flux, and what does that imply about the role of the Dungey cycle in its magnetosphere?
    
    \item Why is the degree of magnetotail hinging insensitive to solar wind dynamic pressure, as seen in the MHD simulations? 
\end{enumerate}