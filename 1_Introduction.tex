% This chapter will introduce Jupiter's magnetosphere to the reader:
% ------------------------------------------------------------------
\chapter{Introduction}

\section{Magnetic reconnection}
Magnetic reconnection is a process which occurs predominantly between regions with oppositely directed magnetic fields, or large magnetic shear. It re-configures the magnetic field and transfers the magnetic energy in the stressed configuration to the particles. Magnetic reconnection is an important ubiquitous phenomena in space plasmas. It is seen to occur on the surface of the sun and in magnetospheres of magnetized planets such as that of the Earth, Mercury, Jupiter, Saturn, Uranus but also in the case of un-magnetized planets such as Venus and Mars. In the terrestrial magnetosphere, magnetic reconnection occurs at the dayside 

\section{Jupiter's magnetosphere}
The interaction of Jupiter's internal magnetic field with the solar wind and interplanetary magnetic field creates its magnetosphere. Jupiter's internal field is the strongest out of all magnetized planets in the solar system, which contributes to the large size for the magnetosphere. Much like at 1 AU, the solar wind is both supersonic and super-Alfvenic at Jupiter's orbit, leading to the formation of a strong bow shock on the dayside. Downstream of the shock lies a region of heated solar wind called the magnetosheath, which is separated from the magnetosphere by another discontinuity called the magnetopause. The magnetopause is a current layer that serves as the boundary separating the magnetic field lines of the interplanetary magnetic field from those connected with the planet. 

\subsection{The inner magnetosphere (\texorpdfstring{$r < 15 R_J$)}{r<15}}
The large natural satellites of the gas giants greatly contribute to plasma dynamics in their magnetospheres. At Jupiter, the largest moons in increasing order of radial distance from the planet are Io, Europa and Ganymede. Out of these, Io and Europa and considered to be un-magnetized, but interact with the surrounding magnetospheric plasma due to the resistive effects of subsurface magma and saline oceans respectively. Ganymede is the only known natural satellite in the solar system which possess a strong internal magnetic field, which interacts with the Jovian magnetic field to create its own magnetosphere. 

Out of these satellites, Io and Europa are responsible for contributing substantial mass to the inner magnetosphere of Jupiter. Volcanism at Io creates SO$_2$, which is ionized either via photo-ionization or electron-impact ionization to produce S$^{++}$, O$^+$ etc. Meanwhile, plumes at Europa eject water neutrals into the surroundings, which ionize to produce water group ions - H$^+$, O$^+$ etc. the net contribution of this ionization results in a mass addition of approximately 250-1000 kg/s for Io and 50 kg/s for Europa. A similar situation is seen at Saturn, where Enceladus adds $\sim$50 kg/s to its magnetosphere. Local loss mechanisms at such as charge exchange cannot fully account for this addition of mass. 

The newly created ions are ``picked-up'' by the surrounding magnetospheric plasma. Newly created plasma ions, which presumably still possess the Keplerian velocity of the neutrals ($\sim$10 km/s at Io's orbit), perceive the motional electric field ($\mathbf{E} = -\mathbf{u} \times \mathbf{B}$) due to the faster magnetospheric flow ($u=\sim200$ km/s) and undergo the $\mathbf{E} \times \mathbf{B}$ drift. Ion pickup increases particle velocity in the direction perpendicular to the magnetic field, increasing anisotropy. 

The magnetospheric plasma in the inner magnetosphere corotates with the planet. Corotation at these distances is facilitated through ion-neutral collisions in the Jovian ionosphere, which prevents a velocity gradient from forming between the thermosphere and ionosphere. The ionosphere then transmits the velocity to all regions in the magnetosphere with which it is connected via magnetic field lines. The corotation process in the inner magnetosphere does not require the presence of large-scale field aligned currents as in the case of the middle and outer magnetosphere, which we will discuss in the next section. 

\subsection{The middle magnetosphere (\texorpdfstring{$15 R_J < r < ~30 R_J$)}{15<r<30}}
The plasma created in the inner magnetosphere of Jupiter is believed to be lost through a series of instabilities. The large density in the inner magnetosphere leads to a decrease in flux tube content with radial distance, which, in the presence of the centrifugal force, creates regions unstable to the interchange instability. Observations by the Galileo and Cassini spacecraft have detected plasma and magnetic signatures containing pockets of high magnetic field strength or high energy plasma which is lower in density, within regions of low energy, high-density plasma. 

It is believed that as Iogenic plasma moves from the inner magnetosphere to further radial distances, the conservation of angular momentum results in the loss of its azimuthal velocity. The slowing down of magnetospheric plasma creates a ``bend-back'' of the frozen-in magnetic field lines in the equatorial plane. The bending of these field lines, or more accurately, the production of magnetic curvature, creates radial currents in the equatorial region to counter the deceleration due to angular-momentum conservation. The radial currents, together with the predominantly southward magnetic field, create a $\mathbf{J}\times\mathbf{B}$ force in the azimuthal direction. These ``corotation-enforcement'' currents are closed via field-aligned currents and perpendicular currents in the Jovian ionosphere. The ionospheric location corresponding to the outward currents (and hence, precipitating electrons) is considered to be the location of the main oval of the Jovian ultraviolet aurora.

The centrifugal force in Jupiter's magnetosphere also stresses the field in the radial direction, leading to the formation of a `magnetodisc' configuration and a strong cross tail current sheet. Evidence for this process has been provided by in-situ spacecraft such as Galileo and Juno, which observed that field lines located at large distances departed greatly from the dipole expectation. 

\subsection{The outer magnetosphere ( \texorpdfstring{$r > 30 R_J$)}{r>30}}
Eventually, the stretching of magnetic field lines at large radial distances on the nightside thins the equatorial current sheet to an extent that allows for magnetic reconnection to take place. Reconnection in Jupiter's magnetosphere is different from the Dungey-cycle reconnection seen in the terrestrial magnetosphere, where open field lines in the magnetotail reconnect and produce a closed field line. At Jupiter, magnetotail reconnection is believed to occur spontaneously due to the thinning of the magnetotail current sheet, leading to reconnection within the same field line. This process detaches a loop-like magnetic structure which is disconnected from the other closed field lines and is free to travel tailward and escape the magnetosphere. Since these loop-like structures transport plasma away from the magnetosphere, they are referred to as ``plasmoids''. The newly reconnected closed field line, now devoid of heavy mass, returns to the planet and continues to corotate and get loaded with mass from the inner magnetosphere. This cycle of mass loss through internal magnetic reconnection is called the ``Vasyliunas'' cycle. The Vasyliunas cycle is considered to be the final process that facilitates the loss of Iogenic plasma from the magnetosphere to the external solar wind.

In-situ observations by Galileo and Juno have shown that plasmoids are created in the Jovian magnetosphere due to magnetic reconnection. The observed plasmoids were fairly large, with diameters ranging between 2-20 $R_J$. Despite their size, it was estimated that the mass carried by theses plasmoids could account for an effective loss of ~200 kg/s, much less than the ~1000 kg/s added by Io. 


\section{Solar wind conditions at Jupiter}
\section{Jupiter's UV aurora}
\section{Jupiter's internal magnetic field}

% \section{Open questions}
% \subsection{Variability of the Jovian UV aurora: Effect of solar wind dynamic pressure?}
% \subsection{Does Dungey-cycle reconnection play an important role?}
% \subsection{How is plasma }

